% Options for packages loaded elsewhere
\PassOptionsToPackage{unicode}{hyperref}
\PassOptionsToPackage{hyphens}{url}
\PassOptionsToPackage{dvipsnames,svgnames,x11names}{xcolor}
%
\documentclass[
  letterpaper,
]{article}

\usepackage{amsmath,amssymb}
\usepackage{iftex}
\ifPDFTeX
  \usepackage[T1]{fontenc}
  \usepackage[utf8]{inputenc}
  \usepackage{textcomp} % provide euro and other symbols
\else % if luatex or xetex
  \usepackage{unicode-math}
  \defaultfontfeatures{Scale=MatchLowercase}
  \defaultfontfeatures[\rmfamily]{Ligatures=TeX,Scale=1}
\fi
\usepackage{lmodern}
\ifPDFTeX\else  
    % xetex/luatex font selection
    \setmainfont[]{Liberation Serif}
    \setmonofont[Scale=0.8]{JetBrainsMono Nerd Font}
\fi
% Use upquote if available, for straight quotes in verbatim environments
\IfFileExists{upquote.sty}{\usepackage{upquote}}{}
\IfFileExists{microtype.sty}{% use microtype if available
  \usepackage[]{microtype}
  \UseMicrotypeSet[protrusion]{basicmath} % disable protrusion for tt fonts
}{}
\makeatletter
\@ifundefined{KOMAClassName}{% if non-KOMA class
  \IfFileExists{parskip.sty}{%
    \usepackage{parskip}
  }{% else
    \setlength{\parindent}{0pt}
    \setlength{\parskip}{6pt plus 2pt minus 1pt}}
}{% if KOMA class
  \KOMAoptions{parskip=half}}
\makeatother
\usepackage{xcolor}
\setlength{\emergencystretch}{3em} % prevent overfull lines
\setcounter{secnumdepth}{5}
% Make \paragraph and \subparagraph free-standing
\makeatletter
\ifx\paragraph\undefined\else
  \let\oldparagraph\paragraph
  \renewcommand{\paragraph}{
    \@ifstar
      \xxxParagraphStar
      \xxxParagraphNoStar
  }
  \newcommand{\xxxParagraphStar}[1]{\oldparagraph*{#1}\mbox{}}
  \newcommand{\xxxParagraphNoStar}[1]{\oldparagraph{#1}\mbox{}}
\fi
\ifx\subparagraph\undefined\else
  \let\oldsubparagraph\subparagraph
  \renewcommand{\subparagraph}{
    \@ifstar
      \xxxSubParagraphStar
      \xxxSubParagraphNoStar
  }
  \newcommand{\xxxSubParagraphStar}[1]{\oldsubparagraph*{#1}\mbox{}}
  \newcommand{\xxxSubParagraphNoStar}[1]{\oldsubparagraph{#1}\mbox{}}
\fi
\makeatother

\usepackage{color}
\usepackage{fancyvrb}
\newcommand{\VerbBar}{|}
\newcommand{\VERB}{\Verb[commandchars=\\\{\}]}
\DefineVerbatimEnvironment{Highlighting}{Verbatim}{commandchars=\\\{\}}
% Add ',fontsize=\small' for more characters per line
\usepackage{framed}
\definecolor{shadecolor}{RGB}{241,243,245}
\newenvironment{Shaded}{\begin{snugshade}}{\end{snugshade}}
\newcommand{\AlertTok}[1]{\textcolor[rgb]{0.68,0.00,0.00}{#1}}
\newcommand{\AnnotationTok}[1]{\textcolor[rgb]{0.37,0.37,0.37}{#1}}
\newcommand{\AttributeTok}[1]{\textcolor[rgb]{0.40,0.45,0.13}{#1}}
\newcommand{\BaseNTok}[1]{\textcolor[rgb]{0.68,0.00,0.00}{#1}}
\newcommand{\BuiltInTok}[1]{\textcolor[rgb]{0.00,0.23,0.31}{#1}}
\newcommand{\CharTok}[1]{\textcolor[rgb]{0.13,0.47,0.30}{#1}}
\newcommand{\CommentTok}[1]{\textcolor[rgb]{0.37,0.37,0.37}{#1}}
\newcommand{\CommentVarTok}[1]{\textcolor[rgb]{0.37,0.37,0.37}{\textit{#1}}}
\newcommand{\ConstantTok}[1]{\textcolor[rgb]{0.56,0.35,0.01}{#1}}
\newcommand{\ControlFlowTok}[1]{\textcolor[rgb]{0.00,0.23,0.31}{\textbf{#1}}}
\newcommand{\DataTypeTok}[1]{\textcolor[rgb]{0.68,0.00,0.00}{#1}}
\newcommand{\DecValTok}[1]{\textcolor[rgb]{0.68,0.00,0.00}{#1}}
\newcommand{\DocumentationTok}[1]{\textcolor[rgb]{0.37,0.37,0.37}{\textit{#1}}}
\newcommand{\ErrorTok}[1]{\textcolor[rgb]{0.68,0.00,0.00}{#1}}
\newcommand{\ExtensionTok}[1]{\textcolor[rgb]{0.00,0.23,0.31}{#1}}
\newcommand{\FloatTok}[1]{\textcolor[rgb]{0.68,0.00,0.00}{#1}}
\newcommand{\FunctionTok}[1]{\textcolor[rgb]{0.28,0.35,0.67}{#1}}
\newcommand{\ImportTok}[1]{\textcolor[rgb]{0.00,0.46,0.62}{#1}}
\newcommand{\InformationTok}[1]{\textcolor[rgb]{0.37,0.37,0.37}{#1}}
\newcommand{\KeywordTok}[1]{\textcolor[rgb]{0.00,0.23,0.31}{\textbf{#1}}}
\newcommand{\NormalTok}[1]{\textcolor[rgb]{0.00,0.23,0.31}{#1}}
\newcommand{\OperatorTok}[1]{\textcolor[rgb]{0.37,0.37,0.37}{#1}}
\newcommand{\OtherTok}[1]{\textcolor[rgb]{0.00,0.23,0.31}{#1}}
\newcommand{\PreprocessorTok}[1]{\textcolor[rgb]{0.68,0.00,0.00}{#1}}
\newcommand{\RegionMarkerTok}[1]{\textcolor[rgb]{0.00,0.23,0.31}{#1}}
\newcommand{\SpecialCharTok}[1]{\textcolor[rgb]{0.37,0.37,0.37}{#1}}
\newcommand{\SpecialStringTok}[1]{\textcolor[rgb]{0.13,0.47,0.30}{#1}}
\newcommand{\StringTok}[1]{\textcolor[rgb]{0.13,0.47,0.30}{#1}}
\newcommand{\VariableTok}[1]{\textcolor[rgb]{0.07,0.07,0.07}{#1}}
\newcommand{\VerbatimStringTok}[1]{\textcolor[rgb]{0.13,0.47,0.30}{#1}}
\newcommand{\WarningTok}[1]{\textcolor[rgb]{0.37,0.37,0.37}{\textit{#1}}}

\providecommand{\tightlist}{%
  \setlength{\itemsep}{0pt}\setlength{\parskip}{0pt}}\usepackage{longtable,booktabs,array}
\usepackage{calc} % for calculating minipage widths
% Correct order of tables after \paragraph or \subparagraph
\usepackage{etoolbox}
\makeatletter
\patchcmd\longtable{\par}{\if@noskipsec\mbox{}\fi\par}{}{}
\makeatother
% Allow footnotes in longtable head/foot
\IfFileExists{footnotehyper.sty}{\usepackage{footnotehyper}}{\usepackage{footnote}}
\makesavenoteenv{longtable}
\usepackage{graphicx}
\makeatletter
\def\maxwidth{\ifdim\Gin@nat@width>\linewidth\linewidth\else\Gin@nat@width\fi}
\def\maxheight{\ifdim\Gin@nat@height>\textheight\textheight\else\Gin@nat@height\fi}
\makeatother
% Scale images if necessary, so that they will not overflow the page
% margins by default, and it is still possible to overwrite the defaults
% using explicit options in \includegraphics[width, height, ...]{}
\setkeys{Gin}{width=\maxwidth,height=\maxheight,keepaspectratio}
% Set default figure placement to htbp
\makeatletter
\def\fps@figure{htbp}
\makeatother

\newcommand{\fontmydefault}{\fontspec{Liberation Serif}}
\newcommand{\fontlibertsan}[1]{{\fontspec{Liberation Sans} #1}}
\newcommand{\fontlibertsnn}[1]{{\fontspec{Liberation Sans Narrow} #1}}
\newcommand{\fontnimbussan}[1]{{\fontspec{Nimbus Sans} #1}}
\newcommand{\fontnimbussnn}[1]{{\fontspec{Nimbus Sans Narrow} #1}}
\newcommand{\fontdejavusan}[1]{{\fontspec{DejaVu Sans} #1}}
\newcommand{\fontdejavuser}[1]{{\fontspec{DejaVu Serif} #1}}
\newcommand{\myfontchart}[1]{{\fontfamily{bch}\selectfont #1}}
\usepackage{colortbl}
\usepackage{multicol}
\newenvironment{thintabl}
  {
    \renewcommand{\arraystretch}{1}     % Row height
    \setlength{\arrayrulewidth}{1.5pt}  % Line thickness
    \setlength{\tabcolsep}{0pt}         % Column padding
    \tabular    % Start regular tabular environment
  }
  {
    \endtabular  % End regular tabular environment
  }
\usepackage{eqparbox}
\makeatletter % This is used to change the behavior of '@' in code below
\NewDocumentCommand{\eqmathbox}{o O{c} m}{ % This allows \eqparbox to be
  \IfValueTF{#1}                           % used inside (math) equations
    {\def\eqmathbox@##1##2{\eqmakebox[#1][#2]{$##1##2$}}}
    {\def\eqmathbox@##1##2{\eqmakebox{$##1##2$}}}
  \mathpalette\eqmathbox@{#3}
}
\makeatother
\usepackage{wrapfig}
\usepackage{array}
\usepackage{float}
\usepackage{booktabs}
\usepackage{arydshln}
\usepackage{multirow}
\usepackage{multicol}
\usepackage{xcolor}
\usepackage{tcolorbox}
\usepackage{geometry}
\usepackage{amsmath}
\geometry{ left=0.5in, right=0.5in, top=0.5in, bottom=0.5in}
\definecolor{pythoncol}{rgb}{0.188,0.412,0.596}
\definecolor{rcol}{rgb}{0.121,0.466,0.705}
\definecolor{bashcol}{rgb}{0.207,0.262,0.392}
\newtcolorbox{pythonheader}{ colback=pythoncol!70, colframe=pythoncol, fontupper=\footnotesize\bfseries\color{white}, boxrule=0.1mm, arc=4mm, left=1.5mm, right=1.5mm, top=0.5mm, bottom=0.5mm, halign=right, sharp corners=south}
\newtcolorbox{rheader}{ colback=rcol!70, colframe=rcol, fontupper=\footnotesize\bfseries\color{white}, boxrule=0.1mm, arc=4mm, left=1.5mm, right=1.5mm, top=0.5mm, bottom=0.5mm, halign=right, sharp corners=south}
\newtcolorbox{bashheader}{ colback=bashcol!70, colframe=bashcol, fontupper=\footnotesize\bfseries\color{white}, boxrule=0.1mm, arc=4mm, left=1.5mm, right=1.5mm, top=0.5mm, bottom=0.5mm, halign=right, sharp corners=south}
\newcolumntype{L}[1]{>{\raggedright\arraybackslash}p{#1}}
\newcolumntype{C}[1]{>{\centering\arraybackslash}p{#1}}
\newcolumntype{M}[1]{>{\centering\arraybackslash}m{#1}}
\newcolumntype{R}[1]{>{\raggedleft\arraybackslash}p{#1}}
\makeatletter
\@ifpackageloaded{caption}{}{\usepackage{caption}}
\AtBeginDocument{%
\ifdefined\contentsname
  \renewcommand*\contentsname{Table of contents}
\else
  \newcommand\contentsname{Table of contents}
\fi
\ifdefined\listfigurename
  \renewcommand*\listfigurename{List of Figures}
\else
  \newcommand\listfigurename{List of Figures}
\fi
\ifdefined\listtablename
  \renewcommand*\listtablename{List of Tables}
\else
  \newcommand\listtablename{List of Tables}
\fi
\ifdefined\figurename
  \renewcommand*\figurename{Figure}
\else
  \newcommand\figurename{Figure}
\fi
\ifdefined\tablename
  \renewcommand*\tablename{Table}
\else
  \newcommand\tablename{Table}
\fi
}
\@ifpackageloaded{float}{}{\usepackage{float}}
\floatstyle{ruled}
\@ifundefined{c@chapter}{\newfloat{codelisting}{h}{lop}}{\newfloat{codelisting}{h}{lop}[chapter]}
\floatname{codelisting}{Listing}
\newcommand*\listoflistings{\listof{codelisting}{List of Listings}}
\makeatother
\makeatletter
\makeatother
\makeatletter
\@ifpackageloaded{caption}{}{\usepackage{caption}}
\@ifpackageloaded{subcaption}{}{\usepackage{subcaption}}
\makeatother

\ifLuaTeX
  \usepackage{selnolig}  % disable illegal ligatures
\fi
\usepackage{bookmark}

\IfFileExists{xurl.sty}{\usepackage{xurl}}{} % add URL line breaks if available
\urlstyle{same} % disable monospaced font for URLs
\hypersetup{
  colorlinks=true,
  linkcolor={blue},
  filecolor={Maroon},
  citecolor={Blue},
  urlcolor={Blue},
  pdfcreator={LaTeX via pandoc}}


\author{}
\date{}

\begin{document}

\begin{titlepage}
\begin{flushleft}
  { UBMI-IFC, UNAM } \\
  { Coyoacan, CDMX }
\end{flushleft}
\vspace*{3cm}
\begin{center}
  { \Large Sequece Characterization Test } \\[1cm]
  { \large Using Genomic-Benchmarks Data } \\[1.5cm]
  { \fontlibertsan{\textbf{THEORY \& CODE ANNEX}} }
\end{center}
\vfill
\begin{flushright}
\begin{tabular}{r@{:\hspace*{\tabcolsep}}l}
  Author & \parbox[t]{2.4cm}{\raggedleft Fuentes-Mendez \\ David Gregorio} \\
\end{tabular}
\end{flushright}
\end{titlepage}

\renewcommand*\contentsname{Table of contents}
{
\hypersetup{linkcolor=}
\setcounter{tocdepth}{3}
\tableofcontents
}

\setlength\parindent{18pt}
\setlength\columnsep{18pt}
\newpage
\twocolumn

\section{Data Recollection}\label{data-recollection}

\subsection{Data Sources and Sequence
Lengths}\label{data-sources-and-sequence-lengths}

\subsubsection{Promoter Elements: Core, Proximal and
Distal}\label{promoter-elements-core-proximal-and-distal}

\subsubsection{Enhancer Elements:}\label{enhancer-elements}

\subsection{Isolation \& Delimitation
Difficulties}\label{isolation-delimitation-difficulties}

\subsection{Promoter Bashing \& Enhancer
Trapping}\label{promoter-bashing-enhancer-trapping}

\section{Determining Sequence
Characterization}\label{determining-sequence-characterization}

\subsection{Key Concepts}\label{key-concepts}

Biological sequences often contain dependencies, such as motifs, repeat
regions, or structural constraints, that influence nucleotide placement.

\begin{rheader}
R Code
\end{rheader}

\begin{Shaded}
\begin{Highlighting}[]
\NormalTok{uniq\_values }\OtherTok{\textless{}{-}} \ControlFlowTok{function}\NormalTok{(vect, }\AttributeTok{round\_digits =} \DecValTok{4}\NormalTok{)\{}
  \FunctionTok{return}\NormalTok{(}\FunctionTok{round}\NormalTok{(}\FunctionTok{as.numeric}\NormalTok{(}
                 \FunctionTok{names}\NormalTok{(}\FunctionTok{table}\NormalTok{(vect))}
\NormalTok{         ), }\AttributeTok{digits =}\NormalTok{ round\_digits))\}}
\NormalTok{uniq\_not }\OtherTok{\textless{}{-}} \ControlFlowTok{function}\NormalTok{(vect, }\AttributeTok{round\_digits=}\DecValTok{4}\NormalTok{)\{}
  \FunctionTok{return}\NormalTok{(}\FunctionTok{round}\NormalTok{(}\FunctionTok{unique}\NormalTok{(}\FunctionTok{sort}\NormalTok{(vect)), }\AttributeTok{digits=}\NormalTok{round\_digits))\}}
\CommentTok{\# \^{}For some reason returned duplicated values when }
\CommentTok{\#  processing floating values from "gc{-}percentage"}
\end{Highlighting}
\end{Shaded}

\subsection{Whole-Sequence
Characterization}\label{whole-sequence-characterization}

\subsection{Kmer Characterization}\label{kmer-characterization}

\begin{Shaded}
\begin{Highlighting}[]
\NormalTok{all\_k3 }\OtherTok{\textless{}{-}} \FunctionTok{combi\_kmers}\NormalTok{(}\AttributeTok{k =} \DecValTok{3}\NormalTok{)}
\NormalTok{all\_k4 }\OtherTok{\textless{}{-}} \FunctionTok{combi\_kmers}\NormalTok{(}\AttributeTok{k =} \DecValTok{4}\NormalTok{)}
\NormalTok{all\_k5 }\OtherTok{\textless{}{-}} \FunctionTok{combi\_kmers}\NormalTok{(}\AttributeTok{k =} \DecValTok{5}\NormalTok{)}
\NormalTok{all\_k6 }\OtherTok{\textless{}{-}} \FunctionTok{combi\_kmers}\NormalTok{(}\AttributeTok{k =} \DecValTok{6}\NormalTok{)}

\FunctionTok{length}\NormalTok{(all\_k3);}\FunctionTok{length}\NormalTok{(all\_k4)}
\FunctionTok{length}\NormalTok{(all\_k5);}\FunctionTok{length}\NormalTok{(all\_k6)}
\end{Highlighting}
\end{Shaded}

\begin{figure}

\begin{minipage}{0.25\linewidth}

\begin{verbatim}
[1] 64
\end{verbatim}

\end{minipage}%
%
\begin{minipage}{0.25\linewidth}

\begin{verbatim}
[1] 256
\end{verbatim}

\end{minipage}%
%
\begin{minipage}{0.25\linewidth}

\begin{verbatim}
[1] 1024
\end{verbatim}

\end{minipage}%
%
\begin{minipage}{0.25\linewidth}

\begin{verbatim}
[1] 4096
\end{verbatim}

\end{minipage}%

\end{figure}%

\subsection{Kmer Distribution
Characterization}\label{kmer-distribution-characterization}

\subsection{Characterization Dependence on Sequence
Length}\label{characterization-dependence-on-sequence-length}

\section{\texorpdfstring{Understanding:\newline GC Percentage \& Melting
Temperature}{Understanding:GC Percentage \& Melting Temperature}}\label{understandinggc-percentage-melting-temperature}

\subsection{GC Percentage: Definition \&
Formula}\label{gc-percentage-definition-formula}

\emph{GC Percentage} (also called GC content) refers to the proportion
of guanine (G) and cytosine (C) bases in a DNA sequence. Since G and C
bases form three hydrogen bonds (as opposed to the two bonds between
adenine (A) and thymine (T) in DNA, sequences with a higher GC content
are typically more stable and harder to denature. It can be calculated
using ths simple formula:

\small

\begin{equation}
GC\% = \frac{(\text{\textbf{C} count} + \text{\textbf{G} count})}{\text{Sequence Length}}
\end{equation} \normalsize

\subsection{Melting Temperature: Definition \&
Formula}\label{melting-temperature-definition-formula}

\emph{Melting Temperature} (Tm) refers to the temperature at DNA
denatures into two separate strands. It is critical, for example, in
experiments like PCR (\emph{Polymerase Chain Reaction}) because the
annealing temperature is usually set a few degrees below the Tm to allow
proper primer binding. It can be estimated using the following formulas:
\vspace{0.08cm}

\begin{itemize}
\tightlist
\item
  \textbf{For short sequences} (3 to 13\textasciitilde20
  nucleotides)\textbf{:} \small\begin{equation}
  Tm=2(\text{\textbf{A} count} + \text{\textbf{T} count}) + 4(\text{\textbf{C} count} + \text{\textbf{G} count})
  \end{equation}\normalsize \vspace{0.04cm}
\item
  \textbf{For long sequences} (\textgreater13\textasciitilde20
  nucleotides)\textbf{:} \small\begin{equation}
  Tm=64.9 + 41 * \frac{(\text{\textbf{A} count} + \text{\textbf{T} count}) - 16.4}{\text{Sequence Length}}
  \end{equation}\normalsize \vspace{0.04cm}
\end{itemize}

\subsection{Issues in Kmer
Characterization}\label{issues-in-kmer-characterization}

\begin{wrapfigure}{r}{4.8cm}
\caption{GC\% Redundancy}

\setlength{\tabcolsep}{0pt}          % Remove column padding
\renewcommand{\arraystretch}{1.65}   % Adjust higher row height 
\setlength{\arrayrulewidth}{1.5pt}   % Set thick frame lines
\centering

\colorbox{gray!20}{\makebox[1cm][c]{\fontlibertsan{\textbf{GC\%:}}}}
\makebox[3.3cm][c]{\fontlibertsan{Example Sequences:}} \\[0.25cm]

\makebox[0.85cm][r]{}
\begin{thintabl}{C{0.79cm} C{0.79cm} C{0.79cm} C{0.79cm}}
    \arrayrulecolor{white}
    \rowcolor{red!10}
    A & A & \cellcolor{red!17}\textbf{G} & T \\  
    \hline
    \rowcolor{red!10}
    T & \cellcolor{red!17}\textbf{C} & A & T \\  
    \hline
    \rowcolor{red!10}
    \cellcolor{red!17}\textbf{G} & A & T & A \\  
    \hline
\end{thintabl} \\[0pt]
\colorbox{gray!20}{\makebox[1cm][c]{\fontlibertsan{\textbf{0.25}}}}
\begin{tabular}{|m{0.75cm}|m{0.75cm}|m{0.75cm}|m{0.75cm}|}
    \hline
    \cellcolor{red!50} & & & \\  
    \hline
\end{tabular} \\[0.25cm]

\makebox[0.85cm][r]{}
\begin{thintabl}{C{0.79cm} C{0.79cm} C{0.79cm} C{0.79cm}}
    \arrayrulecolor{white}
    \rowcolor{red!17}
    \cellcolor{red!10}A & \textbf{G} & \cellcolor{red!10}T & \textbf{C} \\  
    \hline 
    \rowcolor{red!17}
    \cellcolor{red!10}T & \textbf{G} & \textbf{G} & \cellcolor{red!10}T \\  
    \hline
    \rowcolor{red!17}
    \textbf{C} & \textbf{G} & \cellcolor{red!10}A & \cellcolor{red!10}A \\  
    \hline
\end{thintabl} \\[0pt]
\colorbox{gray!20}{\makebox[1cm][c]{\fontlibertsan{\textbf{0.50}}}}
\begin{tabular}{|m{0.75cm}|m{0.75cm}|m{0.75cm}|m{0.75cm}|}
    \hline
    \cellcolor{red!50} & \cellcolor{red!50} & & \\ 
    \hline
\end{tabular} \\[0.25cm]

\makebox[0.85cm][r]{}
\begin{thintabl}{C{0.79cm} C{0.79cm} C{0.79cm} C{0.79cm}}
    \arrayrulecolor{white}
    \rowcolor{red!17}
    \textbf{G} & \cellcolor{red!10}A & \textbf{C} & \textbf{G} \\  
    \hline
    \rowcolor{red!17}
    \textbf{C} & \textbf{C} & \cellcolor{red!10}T & \textbf{C} \\  
    \hline
    \rowcolor{red!17}
    \textbf{C} & \textbf{G} & \textbf{G} & \cellcolor{red!10}A \\  
    \hline
\end{thintabl} \\[0pt]
\colorbox{gray!20}{\makebox[1cm][c]{\fontlibertsan{\textbf{0.75}}}}
\begin{tabular}{|m{0.75cm}|m{0.75cm}|m{0.75cm}|m{0.75cm}|}
    \hline
    \cellcolor{red!50} & \cellcolor{red!50} & \cellcolor{red!50} & \\  
    \hline
\end{tabular}
\end{wrapfigure}

One major issue with GC measurement is its inability to capture the
nucleotide arrangement within a sequence. Two sequences can have
identical GC content but different nucleotide order, leading to
different structural and functional properties
(\fontnimbussnn{\textbf{Figure 1}}). For example
\fontnimbussnn{\textbf{'GCGCTT'}} will evidently behave different than
\fontnimbussnn{\textbf{'GGAACC'}}, even though they have the
\fontnimbussnn{\textbf{same GC\%}} and the
\fontnimbussnn{\textbf{same TM}} (\fontnimbussnn{\textbf{Figure 1}}).

GC ratio alone will fail to capture a complete picture of a sequence's
biological function since it can provide no insight into sequence motifs
or complexity, meaning that it fails to differentiate between truly
complex sequences and those that are repetitive or structurally
constrained. However one thing it can supply is information regarding
sequence length importance. We will notice two opposting issues for each
function: while sequence length does not influence \emph{GC Percentage}
computation, for \emph{Melting Temperature} it does. This should lead to
considerably different experimental approaches dependent on the kind of
data available for our analysis. These approaches were dicussed in
Section 2.5.

Consequently and referent to sequence length, we observe that for
sequences (or kmers) of \textbf{size \emph{N}} we will \textbf{always
obtain at most \emph{N}+1 different values} (considering the empty case
of sequences that lack G/C nucleotides), using either function. A
practical representation of this and the redundancy of values
(considering the amount of kmers), can be observed with the following
code cells: \vspace{0.2cm}

\begin{Shaded}
\begin{Highlighting}[]
\CommentTok{\#List of GC\% values per kmer{-}set}
\NormalTok{all\_k3\_gc }\OtherTok{\textless{}{-}} \FunctionTok{func\_per\_windows}\NormalTok{(}\AttributeTok{windows =}\NormalTok{ all\_k3,}
                              \AttributeTok{func =}\NormalTok{ gc\_percentage)}
\NormalTok{all\_k4\_gc }\OtherTok{\textless{}{-}} \FunctionTok{func\_per\_windows}\NormalTok{(}\AttributeTok{windows =}\NormalTok{ all\_k4,}
                              \AttributeTok{func =}\NormalTok{ gc\_percentage)}
\NormalTok{all\_k5\_gc }\OtherTok{\textless{}{-}} \FunctionTok{func\_per\_windows}\NormalTok{(}\AttributeTok{windows =}\NormalTok{ all\_k5,}
                              \AttributeTok{func =}\NormalTok{ gc\_percentage)}
\CommentTok{\#List of TM values per kmer{-}set}
\NormalTok{all\_k3\_tm }\OtherTok{\textless{}{-}} \FunctionTok{func\_per\_windows}\NormalTok{(}\AttributeTok{windows =}\NormalTok{ all\_k3,}
                              \AttributeTok{func =}\NormalTok{ tm\_calc)}
\NormalTok{all\_k4\_tm }\OtherTok{\textless{}{-}} \FunctionTok{func\_per\_windows}\NormalTok{(}\AttributeTok{windows =}\NormalTok{ all\_k4,}
                              \AttributeTok{func =}\NormalTok{ tm\_calc)}
\NormalTok{all\_k5\_tm }\OtherTok{\textless{}{-}} \FunctionTok{func\_per\_windows}\NormalTok{(}\AttributeTok{windows =}\NormalTok{ all\_k5,}
                              \AttributeTok{func =}\NormalTok{ tm\_calc)}
\NormalTok{all\_k6\_tm }\OtherTok{\textless{}{-}} \FunctionTok{func\_per\_windows}\NormalTok{(}\AttributeTok{windows =}\NormalTok{ all\_k6,}
                              \AttributeTok{func =}\NormalTok{ tm\_calc)}
\end{Highlighting}
\end{Shaded}

\vspace{0.2cm}

\begin{Shaded}
\begin{Highlighting}[]
\NormalTok{uk3\_gc }\OtherTok{\textless{}{-}} \FunctionTok{uniq\_values}\NormalTok{(all\_k3\_gc, }\AttributeTok{round\_digits =} \DecValTok{2}\NormalTok{)}
\NormalTok{uk4\_gc }\OtherTok{\textless{}{-}} \FunctionTok{uniq\_values}\NormalTok{(all\_k4\_gc, }\AttributeTok{round\_digits =} \DecValTok{2}\NormalTok{)}
\NormalTok{uk5\_gc }\OtherTok{\textless{}{-}} \FunctionTok{uniq\_values}\NormalTok{(all\_k5\_gc, }\AttributeTok{round\_digits =} \DecValTok{2}\NormalTok{)}
    
\CommentTok{\# kmer{-}set    | number of     | values \#}
\CommentTok{\# size        | unique values | list   \#}
\FunctionTok{length}\NormalTok{(all\_k3); }\FunctionTok{length}\NormalTok{(uk3\_gc); uk3\_gc}
\FunctionTok{length}\NormalTok{(all\_k4); }\FunctionTok{length}\NormalTok{(uk4\_gc); uk4\_gc}
\FunctionTok{length}\NormalTok{(all\_k5); }\FunctionTok{length}\NormalTok{(uk5\_gc); uk5\_gc}
\end{Highlighting}
\end{Shaded}

\begin{figure}

\begin{minipage}{0.23\linewidth}

\begin{verbatim}
[1] 64
\end{verbatim}

\end{minipage}%
%
\begin{minipage}{0.17\linewidth}

\begin{verbatim}
[1] 4
\end{verbatim}

\end{minipage}%
%
\begin{minipage}{0.60\linewidth}

\begin{verbatim}
[1] 0.00 0.33 0.67 1.00
\end{verbatim}

\end{minipage}%
\newline
\begin{minipage}{0.23\linewidth}

\begin{verbatim}
[1] 256
\end{verbatim}

\end{minipage}%
%
\begin{minipage}{0.17\linewidth}

\begin{verbatim}
[1] 5
\end{verbatim}

\end{minipage}%
%
\begin{minipage}{0.60\linewidth}

\begin{verbatim}
[1] 0.00 0.25 0.50 0.75 1.00
\end{verbatim}

\end{minipage}%
\newline
\begin{minipage}{0.23\linewidth}

\begin{verbatim}
[1] 1024
\end{verbatim}

\end{minipage}%
%
\begin{minipage}{0.17\linewidth}

\begin{verbatim}
[1] 6
\end{verbatim}

\end{minipage}%
%
\begin{minipage}{0.60\linewidth}

\begin{verbatim}
[1] 0.0 0.2 0.4 0.6 0.8 1.0
\end{verbatim}

\end{minipage}%

\end{figure}%

\begin{Shaded}
\begin{Highlighting}[]
\NormalTok{uk3\_tm }\OtherTok{\textless{}{-}} \FunctionTok{unique}\NormalTok{(all\_k3\_tm)}
\NormalTok{uk4\_tm }\OtherTok{\textless{}{-}} \FunctionTok{unique}\NormalTok{(all\_k4\_tm)}
\NormalTok{uk5\_tm }\OtherTok{\textless{}{-}} \FunctionTok{unique}\NormalTok{(all\_k5\_tm)}
\NormalTok{uk6\_tm }\OtherTok{\textless{}{-}} \FunctionTok{unique}\NormalTok{(all\_k6\_tm)}

\FunctionTok{length}\NormalTok{(uk3\_tm); uk3\_tm}
\FunctionTok{length}\NormalTok{(uk4\_tm); uk4\_tm}
\FunctionTok{length}\NormalTok{(uk5\_tm); uk5\_tm}
\FunctionTok{length}\NormalTok{(uk6\_tm); uk6\_tm}
\end{Highlighting}
\end{Shaded}

\begin{figure}

\begin{minipage}{0.20\linewidth}

\begin{verbatim}
[1] 4
\end{verbatim}

\end{minipage}%
%
\begin{minipage}{0.80\linewidth}

\begin{verbatim}
[1]  6  8 10 12
\end{verbatim}

\end{minipage}%
\newline
\begin{minipage}{0.20\linewidth}

\begin{verbatim}
[1] 5
\end{verbatim}

\end{minipage}%
%
\begin{minipage}{0.80\linewidth}

\begin{verbatim}
[1]  8 10 12 14 16
\end{verbatim}

\end{minipage}%
\newline
\begin{minipage}{0.20\linewidth}

\begin{verbatim}
[1] 6
\end{verbatim}

\end{minipage}%
%
\begin{minipage}{0.80\linewidth}

\begin{verbatim}
[1] 10 12 14 16 18 20
\end{verbatim}

\end{minipage}%
\newline
\begin{minipage}{0.20\linewidth}

\begin{verbatim}
[1] 7
\end{verbatim}

\end{minipage}%
%
\begin{minipage}{0.80\linewidth}

\begin{verbatim}
[1] 12 14 16 18 20 22 24
\end{verbatim}

\end{minipage}%

\end{figure}%

In summary, while GC measurement is a useful and simple measure for
basic sequence characterization, its inability to capture nucleotide
arrangement and sequence complexity limits its utility.

\section{\texorpdfstring{Understanding:\newline Shannon Entropy
Coefficient}{Understanding:Shannon Entropy Coefficient}}\label{understandingshannon-entropy-coefficient}

\subsection{Shannon Entropy Coefficient: Definition \&
Formula}\label{shannon-entropy-coefficient-definition-formula}

The \emph{Shannon Entropy Coefficient} is a \emph{measure of
uncertainty} in a probability distribution. If we have a discrete
(countable) random variable \emph{X} with \emph{n} possible outcomes
\{\(x_1, x_2, ..., x_n\)\}, its' Shannon Coefficient would be calculated
by means of multiplying each outcome's probability by its logaritm and
then sum the products; for a system with \emph{n} possible outcomes and
probabilities \(p_1, p_2, ..., p_n\) this formula would be depicted as:

\small

\begin{equation}
H(X) = -\sum_{i=1}^{n} p_i \log_b(p_i)
\end{equation} \normalsize

\textbf{Interpretation:}

\begin{itemize}
\item
  \fontlibertsnn{\underline{\textbf{Minimum Entropy:}}} When there's
  only one possible outcome (i.e.~a group consisting of a single element
  like \{A,A,A,A\}), the entropy value is \emph{zero}.
\item
  \fontlibertsnn{\underline{\textbf{Maximum Entropy:}}} When all
  outcomes are equally likely, the entropy is at its maximum, this
  indicates the highest level of uncertainty. Its value is dependent on
  the logaritm base and the number of possible outcomes:

  \begin{itemize}
  \tightlist
  \item
    In the case of base 2, for a system with \emph{n}
    \fontlibertsan{\underline{equally probable}} outcomes, the maximum
    entropy is \(\log_2(n)\). \fontlibertsan{For example:} \newline

    \begin{itemize}
    \tightlist
    \item
      \fontlibertsnn{\textbf{2 Outcomes:}} \(\log_2(2)=1\) \newline
      i.e.~\{G,C\} or \{A,A,T,T\} \newline
    \item
      \fontlibertsnn{\textbf{4 Outcomes:}} \(\log_2(4)=2\) \newline
      i.e.~\{A,T,G,C\} or \{A,A,T,T,G,G,C,C\} \newline
    \end{itemize}
  \end{itemize}
\end{itemize}

Since we do use \(\log_2(x)\) for our \emph{Shannon Entropy}
calculation, and only the 4 canonical nucleotides are considered for
each sequences characterization, we might as well say that the Shannon
Entropy Coefficient values in our tables are distributed in a spectrum
ranging from \textbf{zero} to \textbf{two} (at most).

\subsection{Issues in Kmer
Characterization}\label{issues-in-kmer-characterization-1}

While this metric provides insights into the complexity and diversity of
nucleotide compositions, through the distribution of nucleotide
frequencies, it suffers from one of the same defects of GC measurement:
\fontnimbussan{it pays no attention to nucleotide positions}.

This becomes fairly important when we consider the influence that
`order' has in a biological context. For instance, even though
\fontnimbussnn{\textbf{'ATG'}} and \fontnimbussnn{\textbf{'TAG'}} are
composed of the same nucleotides, they are a
\fontnimbussnn{\textbf{start}} and \fontnimbussnn{\textbf{stop}} codon,
respectively. Other examples are conserved motifs or regions where the
order of nucleotides determines secondary structure like hairpins,
loops, or binding sites. Furtherfore, given that in regulatory sequences
specific motifs might restrict certain nucleotides to particular
positions (thus, reducing their entropy), this function might be rather
suited as a partial approach to our kmer-characterization objective.

However, I believe the biggest issue here is the lack of distinction
between sequences with \fontnimbussan{
equal diversity but different elements}. Take as an example this two
sequences: \fontnimbussnn{\textbf{'AAAT'
}} and \fontnimbussnn{\textbf{'CGCC'}}, both with identical entropy
values, but clearly different biological properties
(\fontnimbussnn{\textbf{Figure 2}}).

\begin{figure}[h]
\caption{Entropic Redundancy}

\setlength{\tabcolsep}{0pt}          % Remove column padding
\renewcommand{\arraystretch}{1.65}   % Adjust higher row height 
\setlength{\arrayrulewidth}{1.5pt}   % Set thick frame lines
\centering

\colorbox{gray!20}{\makebox[4.25cm][c]{\fontlibertsan{\textbf{Shannon Entropy Coefficient}}}}
\makebox[4.65cm][r]{\fontlibertsan{Example Sequences:}} \\[0.25cm]

\makebox[0.85cm][r]{}
\begin{thintabl}{C{0.79cm} C{0.79cm} C{0.79cm} C{0.79cm}}
    \arrayrulecolor{white}
    \cellcolor{green!20}T & \cellcolor{orange!20}A & \cellcolor{red!20}G & \cellcolor{cyan!20}C \\  
    \hline
    \cellcolor{orange!20}T & \cellcolor{red!20}C & \cellcolor{cyan!20}A & \cellcolor{green!20}G \\  
    \hline
    \cellcolor{red!20}C & \cellcolor{cyan!20}G & \cellcolor{green!20}A & \cellcolor{orange!20}T \\  
    \hline
\end{thintabl}
\makebox[0.975cm][r]{}
\begin{thintabl}{C{0.79cm} C{0.79cm} C{0.79cm} C{0.79cm}}
    \arrayrulecolor{white}
    \rowcolor{green!20}
    A & A & \cellcolor{red!20}T & \cellcolor{cyan!20}C \\  
    \hline
    \rowcolor{green!20}
    \cellcolor{red!20}A & C & C & \cellcolor{cyan!20}T \\  
    \hline
    \rowcolor{green!20}
    \cellcolor{red!20}C & \cellcolor{cyan!20}G & G & G \\  
    \hline
\end{thintabl} \\[0pt]
\colorbox{gray!20}{\makebox[1cm][c]{\fontlibertsan{\textbf{2.00}}}}
\begin{tabular}{|m{0.75cm}|m{0.75cm}|m{0.75cm}|m{0.75cm}|}
    \hline
    \cellcolor{red!50} & \cellcolor{cyan!50} & \cellcolor{green!50} & \cellcolor{orange!50} \\  
    \hline
\end{tabular}
\colorbox{gray!20}{\makebox[1cm][c]{\fontlibertsan{\textbf{1.50}}}}
\begin{tabular}{|m{0.75cm}|m{0.75cm}|m{0.75cm}|m{0.75cm}|}
    \hline
    \cellcolor{red!50} & \cellcolor{cyan!50} & \cellcolor{green!50} & \cellcolor{green!50} \\  
    \hline
\end{tabular} \\[0.25cm]

\makebox[0.85cm][r]{}
\begin{thintabl}{C{0.79cm} C{0.79cm} C{0.79cm} C{0.79cm}}
    \arrayrulecolor{white}
    \cellcolor{red!20}A & \cellcolor{cyan!20}C & \cellcolor{red!20}A & \cellcolor{cyan!20}C \\  
    \hline 
    \cellcolor{red!20}T & \cellcolor{cyan!20}G & \cellcolor{cyan!20}G & \cellcolor{red!20}T \\  
    \hline
    \cellcolor{red!20}T & \cellcolor{red!20}T & \cellcolor{cyan!20}A & \cellcolor{cyan!20}A \\  
    \hline
\end{thintabl} 
\makebox[0.975cm][r]{}
\begin{thintabl}{C{0.79cm} C{0.79cm} C{0.79cm} C{0.79cm}}
    \arrayrulecolor{white}
    \rowcolor{cyan!20}
    A & A & \cellcolor{red!20}T & A \\  
    \hline
    \rowcolor{cyan!20}
    T & \cellcolor{red!20}G & T & T \\  
    \hline
    \rowcolor{cyan!20}
    \cellcolor{red!20}C & G & G & G \\  
    \hline
\end{thintabl} \\[0pt]
\colorbox{gray!20}{\makebox[1cm][c]{\fontlibertsan{\textbf{1.00}}}}
\begin{tabular}{|m{0.75cm}|m{0.75cm}|m{0.75cm}|m{0.75cm}|}
    \hline
    \cellcolor{red!50} & \cellcolor{red!50} & \cellcolor{cyan!50} & \cellcolor{cyan!50} \\ 
    \hline
\end{tabular}
\colorbox{gray!20}{\makebox[1cm][c]{\fontlibertsan{\textbf{0.81}}}}
\begin{tabular}{|m{0.75cm}|m{0.75cm}|m{0.75cm}|m{0.75cm}|}
    \hline
    \cellcolor{red!50} & \cellcolor{cyan!50} & \cellcolor{cyan!50} & \cellcolor{cyan!50} \\  
    \hline
\end{tabular} \\[0.25cm]

\raggedright
\makebox[0.9cm][r]{}
\begin{thintabl}{C{0.79cm} C{0.79cm} C{0.79cm} C{0.79cm}}
    \arrayrulecolor{white}
    \rowcolor{red!20}
    A & A & A & A \\  
    \hline
    \rowcolor{red!20}
    G & G & G & G \\  
    \hline
    \rowcolor{red!20}
    T & T & T & T \\  
    \hline
\end{thintabl} \\[0pt]
\colorbox{gray!20}{\makebox[1cm][c]{\fontlibertsan{\textbf{0.00}}}}
\begin{tabular}{|m{0.75cm}|m{0.75cm}|m{0.75cm}|m{0.75cm}|}
    \hline
    \rowcolor{red!50}
    & & & \\  
    \hline
\end{tabular}
\end{figure}

In order to know the number of possible values per size of kmer-set,
we'll have to use a little bit of \emph{Number Theory}.
In\linebreak particular, \emph{partitions}.

\vspace{0.2cm}

A \underline{partition} of a positive integer \emph{n} is a multiset of
positive integers that sum to \emph{n}. The total number of partitions
of \emph{n} is denoted by \(p_n\). Thus, given that \newline\small  5 =
4 + 1 = 3 + 2 = 3 + 1 + 1 = 2 + 2 + 1 = 2 + 1 + 1 + 1 = 1 + 1 + 1 + 1 +
1 \newline \normalsize\textcolor{white}{--------}is a complete
enumeration of the partitions of 5, \(p(5) = 7\).

\vspace{0.2cm}

There is no simple formula for \(p_n\), however its not hard to find a
generating function for them, one of the examples is in the R library
\emph{`partitions'} (which will be used in the rest of the chapter).
There, the function \fontnimbussan{P()} works as its homonym and
\fontnimbussan{parts()} computes the spread-out matrix of partitions.
Theres also the homolog functions for restricted partitions
\fontnimbussan{R()} and \fontnimbussan{restrictedparts()}, which compute
the number of partitions conditioned upon an \emph{m} maximum number of
parts. \vspace{0.2cm}

\begin{Shaded}
\begin{Highlighting}[]
\FunctionTok{library}\NormalTok{(partitions)}

\FunctionTok{P}\NormalTok{(}\DecValTok{5}\NormalTok{);               }\FunctionTok{R}\NormalTok{(}\DecValTok{4}\NormalTok{,}\DecValTok{5}\NormalTok{, }\AttributeTok{include.zero=}\ConstantTok{TRUE}\NormalTok{);}
\FunctionTok{addpadd}\NormalTok{(}\FunctionTok{parts}\NormalTok{(}\DecValTok{5}\NormalTok{));  }\FunctionTok{restrictedparts}\NormalTok{(}\DecValTok{5}\NormalTok{,}\DecValTok{4}\NormalTok{)}
\end{Highlighting}
\end{Shaded}

\begin{figure}

\begin{minipage}{0.50\linewidth}

\begin{verbatim}
[1] 7
\end{verbatim}

\end{minipage}%
%
\begin{minipage}{0.50\linewidth}

\begin{verbatim}
[1] 6
\end{verbatim}

\end{minipage}%
\newline
\begin{minipage}{0.50\linewidth}

\begin{verbatim}

                   
[1,] 5 4 3 3 2 2 1 
[2,] 0 1 2 1 2 1 1 
[3,] 0 0 0 1 1 1 1 
[4,] 0 0 0 0 0 1 1 
[5,] 0 0 0 0 0 0 1 
    
\end{verbatim}

\end{minipage}%
%
\begin{minipage}{0.50\linewidth}

\begin{verbatim}
                
[1,] 5 4 3 3 2 2
[2,] 0 1 2 1 2 1
[3,] 0 0 0 1 1 1
[4,] 0 0 0 0 0 1
\end{verbatim}

\end{minipage}%

\end{figure}%

\vspace{0.2cm}

Additionally we can code a function like the following: \vspace{0.2cm}

\begin{Shaded}
\begin{Highlighting}[]
\NormalTok{rawpart }\OtherTok{\textless{}{-}} \ControlFlowTok{function}\NormalTok{(n,k)\{}
  \ControlFlowTok{if}\NormalTok{ (}\FunctionTok{missing}\NormalTok{(k)) k }\OtherTok{\textless{}{-}}\NormalTok{ n}
  \ControlFlowTok{if}\NormalTok{ (k }\SpecialCharTok{==} \DecValTok{0}\NormalTok{) }\FunctionTok{return}\NormalTok{(}\DecValTok{0}\NormalTok{)}
  \ControlFlowTok{if}\NormalTok{ (n }\SpecialCharTok{==} \DecValTok{0}\NormalTok{) }\FunctionTok{return}\NormalTok{(}\DecValTok{1}\NormalTok{)}
  \ControlFlowTok{if}\NormalTok{ (n }\SpecialCharTok{\textless{}} \DecValTok{0}\NormalTok{) }\FunctionTok{return}\NormalTok{(}\DecValTok{0}\NormalTok{)}
  \FunctionTok{return}\NormalTok{(}\FunctionTok{rawpart}\NormalTok{(n, k}\DecValTok{{-}1}\NormalTok{) }\SpecialCharTok{+} \FunctionTok{rawpart}\NormalTok{(n}\SpecialCharTok{{-}}\NormalTok{k, k))}
\NormalTok{\}}
\end{Highlighting}
\end{Shaded}

\begin{Shaded}
\begin{Highlighting}[]
\FunctionTok{rawpart}\NormalTok{(}\DecValTok{3}\NormalTok{); }\FunctionTok{rawpart}\NormalTok{(}\DecValTok{4}\NormalTok{); }\FunctionTok{rawpart}\NormalTok{(}\DecValTok{5}\NormalTok{); }\FunctionTok{rawpart}\NormalTok{(}\DecValTok{6}\NormalTok{);}
\FunctionTok{rawpart}\NormalTok{(}\DecValTok{7}\NormalTok{); }\FunctionTok{rawpart}\NormalTok{(}\DecValTok{8}\NormalTok{); }\FunctionTok{rawpart}\NormalTok{(}\DecValTok{9}\NormalTok{); }\FunctionTok{rawpart}\NormalTok{(}\DecValTok{10}\NormalTok{);}
\FunctionTok{hspace}\NormalTok{();}
\FunctionTok{P}\NormalTok{(}\DecValTok{3}\NormalTok{);       }\FunctionTok{P}\NormalTok{(}\DecValTok{4}\NormalTok{);       }\FunctionTok{P}\NormalTok{(}\DecValTok{5}\NormalTok{);       }\FunctionTok{P}\NormalTok{(}\DecValTok{6}\NormalTok{);}
\FunctionTok{P}\NormalTok{(}\DecValTok{7}\NormalTok{);       }\FunctionTok{P}\NormalTok{(}\DecValTok{8}\NormalTok{);       }\FunctionTok{P}\NormalTok{(}\DecValTok{9}\NormalTok{);       }\FunctionTok{P}\NormalTok{(}\DecValTok{10}\NormalTok{);}
\end{Highlighting}
\end{Shaded}

\begin{figure}

\begin{minipage}{0.25\linewidth}

\begin{verbatim}
[1] 3
\end{verbatim}

\end{minipage}%
%
\begin{minipage}{0.25\linewidth}

\begin{verbatim}
[1] 5
\end{verbatim}

\end{minipage}%
%
\begin{minipage}{0.25\linewidth}

\begin{verbatim}
[1] 7
\end{verbatim}

\end{minipage}%
%
\begin{minipage}{0.25\linewidth}

\begin{verbatim}
[1] 11
\end{verbatim}

\end{minipage}%
\newline
\begin{minipage}{0.25\linewidth}

\begin{verbatim}
[1] 15
\end{verbatim}

\end{minipage}%
%
\begin{minipage}{0.25\linewidth}

\begin{verbatim}
[1] 22
\end{verbatim}

\end{minipage}%
%
\begin{minipage}{0.25\linewidth}

\begin{verbatim}
[1] 30
\end{verbatim}

\end{minipage}%
%
\begin{minipage}{0.25\linewidth}

\begin{verbatim}
[1] 42
\end{verbatim}

\end{minipage}%
\newline
\begin{minipage}{\linewidth}
\textcolor{white}{\tiny\texttt{hi}}\normalsize\end{minipage}%
\newline
\begin{minipage}{0.25\linewidth}

\begin{verbatim}
[1] 3
\end{verbatim}

\end{minipage}%
%
\begin{minipage}{0.25\linewidth}

\begin{verbatim}
[1] 5
\end{verbatim}

\end{minipage}%
%
\begin{minipage}{0.25\linewidth}

\begin{verbatim}
[1] 7
\end{verbatim}

\end{minipage}%
%
\begin{minipage}{0.25\linewidth}

\begin{verbatim}
[1] 11
\end{verbatim}

\end{minipage}%
\newline
\begin{minipage}{0.25\linewidth}

\begin{verbatim}
[1] 15
\end{verbatim}

\end{minipage}%
%
\begin{minipage}{0.25\linewidth}

\begin{verbatim}
[1] 22
\end{verbatim}

\end{minipage}%
%
\begin{minipage}{0.25\linewidth}

\begin{verbatim}
[1] 30
\end{verbatim}

\end{minipage}%
%
\begin{minipage}{0.25\linewidth}

\begin{verbatim}
[1] 42
\end{verbatim}

\end{minipage}%

\end{figure}%

There exist also multiple mathematical formulas for computation. However
the oldest one is Euler's geometric series where the coefficient of
\(x^n\) is equal to \(p_n\), and it can be shown as:

\small

\begin{equation}
\begin{aligned}
    & \eqmathbox[eqn8]{(1+x+x^2+x^3+\dots)(1+x^2+x^4+x^6+\dots)} \\
    & \eqmathbox[eqn8]{(1+x^3+x^6+x^9+\dots)\dots(1+x^k+x^{2k}+x^{3k}+\dots)}
    % \eqmathbox is used to center-align this part of the equation
\end{aligned}
\begin{aligned}
    = \prod_{k=1}^{\infty}\sum_{i=0}^{\infty} x^{ik}
\end{aligned}
\end{equation} \normalsize \vspace{0.25cm}

\begin{Shaded}
\begin{Highlighting}[]
\NormalTok{rparts }\OtherTok{\textless{}{-}}\NormalTok{ restrictedparts}
\NormalTok{Rz }\OtherTok{\textless{}{-}} \ControlFlowTok{function}\NormalTok{(n\_opts, n\_slots) \{}
  \FunctionTok{return}\NormalTok{(}\FunctionTok{R}\NormalTok{(n\_opts, n\_slots, }\AttributeTok{include.zero =} \ConstantTok{TRUE}\NormalTok{))}
\NormalTok{\}}
\NormalTok{rP }\OtherTok{\textless{}{-}} \ControlFlowTok{function}\NormalTok{(n\_opts, n\_slots) \{}
  \FunctionTok{return}\NormalTok{(}\FunctionTok{ncol}\NormalTok{(}\FunctionTok{rparts}\NormalTok{(n\_slots, n\_opts)))}
\NormalTok{\}}
\end{Highlighting}
\end{Shaded}

\vspace{0.33cm}

\begin{Shaded}
\begin{Highlighting}[]
\NormalTok{all\_k3\_sh }\OtherTok{\textless{}{-}} \FunctionTok{func\_per\_windows}\NormalTok{(}\AttributeTok{windows =}\NormalTok{ all\_k3,}
                              \AttributeTok{func =}\NormalTok{ shannon\_entropy)}
\NormalTok{all\_k4\_sh }\OtherTok{\textless{}{-}} \FunctionTok{func\_per\_windows}\NormalTok{(}\AttributeTok{windows =}\NormalTok{ all\_k4,}
                              \AttributeTok{func =}\NormalTok{ shannon\_entropy)}
\NormalTok{all\_k5\_sh }\OtherTok{\textless{}{-}} \FunctionTok{func\_per\_windows}\NormalTok{(}\AttributeTok{windows =}\NormalTok{ all\_k5,}
                              \AttributeTok{func =}\NormalTok{ shannon\_entropy)}
\NormalTok{all\_k6\_sh }\OtherTok{\textless{}{-}} \FunctionTok{func\_per\_windows}\NormalTok{(}\AttributeTok{windows =}\NormalTok{ all\_k6,}
                              \AttributeTok{func =}\NormalTok{ shannon\_entropy)}

\NormalTok{uk3\_sh }\OtherTok{\textless{}{-}} \FunctionTok{uniq\_values}\NormalTok{(all\_k3\_sh, }\AttributeTok{round\_digits =} \DecValTok{2}\NormalTok{)}
\NormalTok{uk4\_sh }\OtherTok{\textless{}{-}} \FunctionTok{uniq\_values}\NormalTok{(all\_k4\_sh, }\AttributeTok{round\_digits =} \DecValTok{2}\NormalTok{)}
\NormalTok{uk5\_sh }\OtherTok{\textless{}{-}} \FunctionTok{uniq\_values}\NormalTok{(all\_k5\_sh, }\AttributeTok{round\_digits =} \DecValTok{2}\NormalTok{)}
\NormalTok{uk6\_sh }\OtherTok{\textless{}{-}} \FunctionTok{uniq\_values}\NormalTok{(all\_k6\_sh, }\AttributeTok{round\_digits =} \DecValTok{2}\NormalTok{)}

\FunctionTok{length}\NormalTok{(uk3\_sh); }\FunctionTok{rP}\NormalTok{(}\DecValTok{4}\NormalTok{,}\DecValTok{3}\NormalTok{); uk3\_sh}
\FunctionTok{length}\NormalTok{(uk4\_sh); }\FunctionTok{rP}\NormalTok{(}\DecValTok{4}\NormalTok{,}\DecValTok{4}\NormalTok{); uk4\_sh}
\FunctionTok{length}\NormalTok{(uk5\_sh); }\FunctionTok{rP}\NormalTok{(}\DecValTok{4}\NormalTok{,}\DecValTok{5}\NormalTok{); uk5\_sh}
\FunctionTok{length}\NormalTok{(uk6\_sh); }\FunctionTok{rP}\NormalTok{(}\DecValTok{4}\NormalTok{,}\DecValTok{6}\NormalTok{); }\FunctionTok{outputwrap}\NormalTok{(uk6\_sh, }\AttributeTok{width =} \DecValTok{30}\NormalTok{)}
\end{Highlighting}
\end{Shaded}

\begin{figure}

\begin{minipage}{0.15\linewidth}

\begin{verbatim}
[1] 3
\end{verbatim}

\end{minipage}%
%
\begin{minipage}{0.15\linewidth}

\begin{verbatim}
[1] 3
\end{verbatim}

\end{minipage}%
%
\begin{minipage}{0.70\linewidth}

\begin{verbatim}
[1] 0.00 0.92 1.58
\end{verbatim}

\end{minipage}%
\newline
\begin{minipage}{0.15\linewidth}

\begin{verbatim}
[1] 5
\end{verbatim}

\end{minipage}%
%
\begin{minipage}{0.15\linewidth}

\begin{verbatim}
[1] 5
\end{verbatim}

\end{minipage}%
%
\begin{minipage}{0.70\linewidth}

\begin{verbatim}
[1] 0.00 0.81 1.00 1.50 2.00
\end{verbatim}

\end{minipage}%
\newline
\begin{minipage}{0.15\linewidth}

\begin{verbatim}
[1] 6
\end{verbatim}

\end{minipage}%
%
\begin{minipage}{0.15\linewidth}

\begin{verbatim}
[1] 6
\end{verbatim}

\end{minipage}%
%
\begin{minipage}{0.70\linewidth}

\begin{verbatim}
[1] 0.00 0.72 0.97 1.37 1.52 1.92
\end{verbatim}

\end{minipage}%
\newline
\begin{minipage}{0.15\linewidth}

\begin{verbatim}
[1] 9
\end{verbatim}

\end{minipage}%
%
\begin{minipage}{0.15\linewidth}

\begin{verbatim}
[1] 9
\end{verbatim}

\end{minipage}%
%
\begin{minipage}{0.70\linewidth}

\begin{verbatim}


[1] 0.00 0.65 0.92 1.00 1.25 1.46 
    1.58 1.79 1.92
\end{verbatim}

\end{minipage}%

\end{figure}%

\begin{Shaded}
\begin{Highlighting}[]
\FunctionTok{rparts}\NormalTok{(}\DecValTok{4}\NormalTok{,}\DecValTok{4}\NormalTok{);     }\FunctionTok{rparts}\NormalTok{(}\DecValTok{5}\NormalTok{,}\DecValTok{4}\NormalTok{);     }\FunctionTok{rparts}\NormalTok{(}\DecValTok{6}\NormalTok{,}\DecValTok{4}\NormalTok{);}
\end{Highlighting}
\end{Shaded}

\begin{figure}

\begin{minipage}{0.28\linewidth}

\begin{verbatim}
              
[1,] 4 3 2 2 1
[2,] 0 1 2 1 1
[3,] 0 0 0 1 1
[4,] 0 0 0 0 1
\end{verbatim}

\end{minipage}%
%
\begin{minipage}{0.32\linewidth}

\begin{verbatim}
                
[1,] 5 4 3 3 2 2
[2,] 0 1 2 1 2 1
[3,] 0 0 0 1 1 1
[4,] 0 0 0 0 0 1
\end{verbatim}

\end{minipage}%
%
\begin{minipage}{0.40\linewidth}

\begin{verbatim}
                      
[1,] 6 5 4 3 4 3 2 3 2
[2,] 0 1 2 3 1 2 2 1 2
[3,] 0 0 0 0 1 1 2 1 1
[4,] 0 0 0 0 0 0 0 1 1
\end{verbatim}

\end{minipage}%

\end{figure}%

\begin{Shaded}
\begin{Highlighting}[]
\FunctionTok{addpadd}\NormalTok{(}\FunctionTok{rparts}\NormalTok{(}\DecValTok{4}\NormalTok{,}\DecValTok{4}\NormalTok{));   }\FunctionTok{addpadd}\NormalTok{(}\FunctionTok{rparts}\NormalTok{(}\DecValTok{5}\NormalTok{,}\DecValTok{4}\NormalTok{)); }
\FunctionTok{addpadd}\NormalTok{(}\FunctionTok{rparts}\NormalTok{(}\DecValTok{6}\NormalTok{,}\DecValTok{4}\NormalTok{));   }\FunctionTok{addpadd}\NormalTok{(}\FunctionTok{rparts}\NormalTok{(}\DecValTok{7}\NormalTok{,}\DecValTok{4}\NormalTok{));}
\end{Highlighting}
\end{Shaded}

\begin{figure}

\begin{minipage}{0.50\linewidth}

\begin{verbatim}

               
[1,] 4 3 2 2 1 
[2,] 0 1 2 1 1 
[3,] 0 0 0 1 1 
[4,] 0 0 0 0 1 
    
\end{verbatim}

\end{minipage}%
%
\begin{minipage}{0.50\linewidth}

\begin{verbatim}

                 
[1,] 5 4 3 3 2 2 
[2,] 0 1 2 1 2 1 
[3,] 0 0 0 1 1 1 
[4,] 0 0 0 0 0 1 
    
\end{verbatim}

\end{minipage}%
\newline
\begin{minipage}{0.50\linewidth}

\begin{verbatim}

                       
[1,] 6 5 4 3 4 3 2 3 2 
[2,] 0 1 2 3 1 2 2 1 2 
[3,] 0 0 0 0 1 1 2 1 1 
[4,] 0 0 0 0 0 0 0 1 1 
    
\end{verbatim}

\end{minipage}%
%
\begin{minipage}{0.50\linewidth}

\begin{verbatim}

                           
[1,] 7 6 5 4 5 4 3 3 4 3 2 
[2,] 0 1 2 3 1 2 3 2 1 2 2 
[3,] 0 0 0 0 1 1 1 2 1 1 2 
[4,] 0 0 0 0 0 0 0 0 1 1 1 
    
\end{verbatim}

\end{minipage}%

\end{figure}%

In summary, while Shannon entropy is a valuable tool for measuring
sequence diversity and randomness, it has significant limitations in
biological sequence characterization. Its assumption of positional
independence and inability to reflect nucleotide order make it less
suitable as a standalone metric for understanding the complexity and
functionality of nucleotide sequences. To gain deeper insights, Shannon
entropy should be used in combination with other metrics that account
for sequence structure, motifs, and functional relevance.

Blablalbalalblab




\end{document}
